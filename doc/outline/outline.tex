\documentclass[a4paper,pagesize]{scrartcl}
\usepackage[utf8]{inputenc}
\usepackage[T1]{fontenc}
\usepackage[english]{babel}

\title{Outline for Master thesis: ``Studies on approximations of Spanning Trees
with Low Crossing Number''}
\author{Maximilian Konzack}
\date{\today}
\begin{document}
	\maketitle
    \begin{enumerate}
        \item Introduction
            \begin{enumerate}
                \item Definiton of the problem
                    \begin{itemize}
                        \item Example graph in the plane
                        \item Crossing distance
                        \item Equivalence relation on line set
                        \item Worst case crossing number
                    \end{itemize}
                \item Generalization to $d$ dimensions
                \item Variations of crossing numbers
                    \begin{itemize}
                        \item Spanning crossing number (minimum crossing
                            number)
                        \item Stabbing number
                    \end{itemize}
                \item Similiar problems
                    \begin{itemize}
                        \item Perfect matching
                        \item Triangulations of minimum total lenght
                            (\emph{non} Steiner ones)
                    \end{itemize}
            \end{enumerate}

        \item Complexity of the problem
            \begin{itemize}
                \item Overview of NP-Hardness
                \item Finding optimum
                \item Integer Program with exponential constraints
            \end{itemize}

        \item Approximation approaches
            \begin{enumerate}
                \item LP relaxation by Fekete
                    \begin{itemize}
                        \item Planarity heuristics
                        \item Iterative rounding scheme
                    \end{itemize}
                \item Multiplicative weights update scheme
                    \begin{itemize}
                        \item Approximation algorithm
                        \item Used facts: crossing distance, crossing disk
                    \end{itemize}
                \item General iterative, LP-based approximation scheme by Sariel
                    \begin{itemize}
                        \item LP formulation with bounded VC dimensions
                        \item Listing of generic approximation algorithm
                        \item Randomized rounding scheme
                        \item Tailoring to $d$ dimensions and planar case
                        \item Deterministic rounding in the plane
                    \end{itemize}
                \item Challenges
                    \begin{itemize}
                        \item Self crossing edges in approximation
                        \item Computing spanning tree within connected
                            components
                        \item \dots
                    \end{itemize}
            \end{enumerate}

        \item New iterative, LP-based Approximation scheme
            \begin{itemize}
                \item Sariel's approach revisited
                \item LP formulation with connected components
                \item Rounding scheme
                \item Listing of the algorithm
            \end{itemize}

        \item Results
            \begin{enumerate}
                \item Computational studies
                    \begin{itemize}
                        \item Problem sets (Grid, Uniform distribution, high
                            dimensional data sets, \dots)
                        \item Implementation details
                        \item Hardware
                    \end{itemize}
                \item Observations on experiments
                    \begin{itemize}
                        \item Pros and Cons of different approximation schemes
                        \item Comparison with Fekete's technical report
                    \end{itemize}
                \item Proofed facts
            \end{enumerate}

        \item Related Work
            \begin{itemize}
                \item Relative crossing number
                \item Overall small crossing number
            \end{itemize}
        \item Conclusion
    \end{enumerate}
\end{document}
